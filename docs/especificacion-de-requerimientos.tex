\chapter{Especificación de Requerimientos}
\section{Requerimientos Funcionales}
\begin{itemize}
  \item Gestión de tareas:
    \begin{itemize}
      \item Crear tareas.
      \item Editar tareas.
      \item Eliminar tareas.
      \item Marcar tareas como completadas.
    \end{itemize}
  \item Visualización de tareas:
    \begin{itemize}
      \item Interfaz intuitiva y práctica de utilizar.
      \item El usuario pueda reconocer todos los cambios realizados de manera visual.
    \end{itemize}
\end{itemize}

\section{Requerimientos No Funcionales}
\begin{itemize}
  \item Rendimiento:
    \begin{itemize}
      \item No debe de haber errores en la logica de programacion.
      \item La aplicación debe de ser rápida y eficiente.
    \end{itemize}
\end{itemize}
\section{Reglas de Negocio}
\begin{enumerate}[start=1, label={RN\arabic*.}]
  \item Las tareas deben de tener un nombre
  \item Las tareas hija no deben de pertenecer a m\'ultiples tareas padre
  \item Las tareas pueden tener una descripci\'on
  \item Las tareas pueden tener una fecha
  \item Las tareas tienen dos estados
  \begin{itemize}
    \item Sin completar
    \item Completado
  \end{itemize}
\end{enumerate}
\section{Casos de Uso}
\begin{center}
\begin{tabular}{p{3cm}p{5cm}p{4cm}p{2cm}}
  \hline
  ID y nombre & CU-1. Crear una tarea & & \\
  \hline
  Creado por: & Alejandro & Fecha de creaci\'on & 24-11-05\\
  \hline
  Actor principal: & Usuario final & Actores secundarios: & Base de datos, aplicaci\'on principal\\
  \hline
  Descripci\'on & \multicolumn{3}{p{11cm}}{El usuario final especificar\'a una tarea a crear con el bot\'on para crear una nueva tarea. Tendr\'a que especificar el nombre de la tarea (obligatorio), una descripci\'on (opcional) y una fecha de la tarea (opcional)}\\
  \hline
  Trigger: & \multicolumn{3}{p{11cm}}{El usuario final indica que quiere crear una tarea con el bot\'on especificado}\\
  \hline
  Precondiciones: & \multicolumn{3}{p{11cm}}{PRE-1. El usuario est\'a registrado en la base de datos}\\
		  & \multicolumn{3}{p{11cm}}{PRE-2. El usuario cuenta con conexi\'on a internet para acceder al sistema}\\
  \hline
  Postcondiciones: & \multicolumn{3}{p{11cm}}{POST-1. La tarea ser\'a guardada en la base de datos}\\
		   & \multicolumn{3}{p{11cm}}{POST-2. La tarea ser\'a mostrada en la p\'agina principal del usuario}\\
  \hline
\end{tabular}
\end{center}


