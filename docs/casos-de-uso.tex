\begin{center}
\begin{tabular}{p{3cm}p{5cm}p{4cm}p{2cm}}
  \hline
  ID y nombre & CU-1. Crear una tarea & & \\
  \hline
  Creado por: & Alejandro & Fecha de creaci\'on & 24-11-05\\
  \hline
  Actor principal: & Usuario final & Actores secundarios: & Base de datos, aplicaci\'on principal\\
  \hline
  Descripci\'on & \multicolumn{3}{p{11cm}}{El usuario final especificar\'a una tarea a crear con el bot\'on para crear una nueva tarea. Tendr\'a que especificar el nombre de la tarea (obligatorio), una descripci\'on (opcional) y una fecha de la tarea (opcional)}\\
  \hline
  Trigger: & \multicolumn{3}{p{11cm}}{El usuario final indica que quiere crear una tarea con el bot\'on especificado}\\
  \hline
  Precondiciones: & \multicolumn{3}{p{11cm}}{PRE-1. El usuario est\'a registrado en la base de datos}\\
		  & \multicolumn{3}{p{11cm}}{PRE-2. El usuario cuenta con conexi\'on a internet para acceder al sistema}\\
  \hline
  Postcondiciones: & \multicolumn{3}{p{11cm}}{POST-1. La tarea ser\'a guardada en la base de datos}\\
		   & \multicolumn{3}{p{11cm}}{POST-2. La tarea ser\'a mostrada en la p\'agina principal del usuario}\\
  \hline
  Flujo normal: & \textbf{1.0. Crear una tarea}\\
		& El usuario final pide al sistema crear una nueva tarea mediante la aplicaci\'on\\
		& La aplicaci\'on despliega los campos a llenar para la tarea\\
		& Una vez llenados los campos, el usuario interact\'ua con la aplicaci\'on para guardar la tarea\\
		& La tarea es enviada a la base de datos del sistema y la valida\\
  \hline
  Excepciones: & \textbf{1.0.E1. La tarea ya ha sido creada}\\
	       & Si la validaci\'on de la base de datos encuentra que la tarea ya ha sido creada, arroja un error \\
	       & La aplicaci\'on escucha si la base de datos arroj\'o el error, y muestra al usuario que la tarea ya ha sido creada\\
	       & El usuario no podr\'a guardar la tarea hasta que el campo del t\'itulo sea cambiado
  \hline
\end{tabular}
\end{center}

